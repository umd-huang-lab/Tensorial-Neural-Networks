\begin{figure}[!htbp]
\vspace{-1em}
\begin{minipage}[b]{.5\linewidth}
\centering
\resizebox{\textwidth}{!}{
   \begin{tikzpicture}

\definecolor{color0}{rgb}{0.886274509803922,0.290196078431373,0.2}
\definecolor{color2}{rgb}{0.203921568627451,0.541176470588235,0.741176470588235}
\definecolor{color1}{rgb}{0.596078431372549,0.556862745098039,0.835294117647059}

\begin{axis}[
axis line style={white},
legend cell align={left},
legend style={
  fill opacity=0.8,
  draw opacity=1,
  text opacity=1,
  at={(0.03, .97)},
  anchor=north west,
  draw=white!80!black
},
tick align=outside,
x grid style={white},
xlabel={compression rate},
xmajorticks=true,
xtick style={draw=none},
xmin=0.7, xmax=7.3,
xtick style={color=white!33.3333333333333!black},
xtick={0,1,2,3,4,5,6,7,8},
xticklabels={,1\%,2\%,5\%,10\%,20\%,50\%,100\%,},
y grid style={white!93.3333333333333!black},
ylabel={seconds per epoch},
ymajorgrids,
ymajorticks=true,
ytick style={draw=none},
ymin=1049.36179748519, ymax=4560.72815214266,
ytick style={color=white!33.3333333333333!black}
]
\path [fill=color0, fill opacity=0.2, very thin]
(axis cs:1,1239.52745899885)
--(axis cs:1,1208.96935906053)
--(axis cs:2,1219.52911242428)
--(axis cs:3,1264.72770568145)
--(axis cs:4,1314.11283386917)
--(axis cs:5,1310.60131838066)
--(axis cs:6,1481.39696673138)
--(axis cs:7,3240.75263688946)
--(axis cs:7,3272.38492714125)
--(axis cs:7,3272.38492714125)
--(axis cs:6,1489.5574229378)
--(axis cs:5,1382.84489640597)
--(axis cs:4,1364.69173989335)
--(axis cs:3,1291.67569619989)
--(axis cs:2,1250.89435414369)
--(axis cs:1,1239.52745899885)
--cycle;

\path [fill=color1, fill opacity=0.2, very thin]
(axis cs:1,1827.25514620111)
--(axis cs:1,1760.39553719911)
--(axis cs:2,1808.7589719549)
--(axis cs:3,2445.03883927714)
--(axis cs:4,4364.01023807436)
--(axis cs:4,4401.12059056732)
--(axis cs:4,4401.12059056732)
--(axis cs:3,2568.05167320782)
--(axis cs:2,1837.95356561557)
--(axis cs:1,1827.25514620111)
--cycle;

\path [fill=color2, fill opacity=0.2, very thin]
(axis cs:1,2913.93184443683)
--(axis cs:1,2848.5072446472)
--(axis cs:1,2913.93184443683)
--(axis cs:1,2913.93184443683)
--cycle;

\addplot [thick, color0, mark=*, mark size=3, mark options={solid}]
table {%
1 1223.9308060529
2 1235.17065172457
3 1277.74342014746
4 1339.26823402275
5 1346.10656586897
6 1485.5358628964
7 3256.12842998683
};
\addlegendentry{\conveinsum}
\addplot [thick, color1, mark=*, mark size=3, mark options={solid}]
table {%
1 1794.62780300269
2 1823.52637473414
3 2507.87693680823
4 4381.0754354547
};
\addlegendentry{\pytorch w/ ckpt}
\addplot [thick, color2, mark=*, mark size=3, mark options={solid}]
table {%
1 2881.26383770777
};
\addlegendentry{\pytorch  w/o ckpt}
\end{axis}

\end{tikzpicture}
    }
\subcaption{RCP Train (spatial)}\label{fig:video-rcp-vs-pytorch-train-spatial}
\end{minipage}%
\hfill
\begin{minipage}[b]{.5\linewidth}
\centering
\resizebox{\textwidth}{!}{
    % This file was created by tikzplotlib v0.9.8.
\begin{tikzpicture}

\definecolor{color0}{rgb}{0.886274509803922,0.290196078431373,0.2}
\definecolor{color2}{rgb}{0.203921568627451,0.541176470588235,0.741176470588235}
\definecolor{color1}{rgb}{0.596078431372549,0.556862745098039,0.835294117647059}

\begin{axis}[
axis line style={white},
legend cell align={left},
legend style={
  fill opacity=0.8,
  draw opacity=1,
  text opacity=1,
  at={(0.03, .97)},
  anchor=north west,
  draw=white!80!black
},
tick align=outside,
x grid style={white},
xlabel={compression rate},
xmajorticks=true,
xtick style={draw=none},
xmin=0.7, xmax=7.3,
xtick style={color=white!33.3333333333333!black},
xtick={0,1,2,3,4,5,6,7,8},
xticklabels={,1\%,2\%,5\%,10\%,20\%,50\%,100\%,},
y grid style={white!93.3333333333333!black},
ylabel={seconds per epoch},
ymajorgrids,
ymajorticks=true,
ytick style={draw=none},
ymin=569.971466073411, ymax=4308.63072954234,
ytick style={color=white!33.3333333333333!black}
]
\path [fill=color0, fill opacity=0.2, very thin]
(axis cs:1,762.306239838883)
--(axis cs:1,739.910523503816)
--(axis cs:2,763.074564933797)
--(axis cs:3,789.492653098703)
--(axis cs:4,887.874895808931)
--(axis cs:5,972.644076023388)
--(axis cs:6,1304.93861951356)
--(axis cs:7,4122.0327110838)
--(axis cs:7,4138.69167211193)
--(axis cs:7,4138.69167211193)
--(axis cs:6,1320.86505331794)
--(axis cs:5,995.159153544717)
--(axis cs:4,890.361982982634)
--(axis cs:3,803.373229607015)
--(axis cs:2,777.384986915317)
--(axis cs:1,762.306239838883)
--cycle;

\path [fill=color1, fill opacity=0.2, very thin]
(axis cs:1,1496.55048084937)
--(axis cs:1,1457.65330565694)
--(axis cs:2,1530.22273459423)
--(axis cs:3,1844.85296048969)
--(axis cs:4,2707.86924717783)
--(axis cs:4,2719.43859059785)
--(axis cs:4,2719.43859059785)
--(axis cs:3,1856.21068725365)
--(axis cs:2,1577.76604877951)
--(axis cs:1,1496.55048084937)
--cycle;

\path [fill=color2, fill opacity=0.2, very thin]
(axis cs:1,2612.0985164882)
--(axis cs:1,2597.19022640012)
--(axis cs:1,2612.0985164882)
--(axis cs:1,2612.0985164882)
--cycle;

\addplot [thick, color0, mark=*, mark size=3, mark options={solid}]
table {%
1 750.883967776807
2 770.309357175781
3 796.727914231485
4 889.107208063332
5 983.561759136815
6 1312.90264605939
7 4130.13177181716
};
\addlegendentry{\conveinsum}
\addplot [thick, color1, mark=*, mark size=3, mark options={solid}]
table {%
1 1477.65929167555
2 1554.07095801219
3 1850.86765552539
4 2713.68764614659
};
\addlegendentry{\pytorch w/ ckpt}
\addplot [thick, color2, mark=*, mark size=3, mark options={solid}]
table {%
1 2604.73291728878
};
\addlegendentry{\pytorch  w/o ckpt}
\end{axis}

\end{tikzpicture}
    }
\subcaption{RCP Test (spatial)}\label{fig:video-rcp-vs-pytorch-test-spatial}
\end{minipage}
\begin{minipage}[b]{.5\linewidth}
\centering
\resizebox{\textwidth}{!}{
   \begin{tikzpicture}

\definecolor{color0}{rgb}{0.886274509803922,0.290196078431373,0.2}
\definecolor{color2}{rgb}{0.203921568627451,0.541176470588235,0.741176470588235}
\definecolor{color1}{rgb}{0.596078431372549,0.556862745098039,0.835294117647059}

\begin{axis}[
axis line style={white},
legend cell align={left},
legend style={
  fill opacity=0.8,
  draw opacity=1,
  text opacity=1,
  at={(0.03, .97)},
  anchor=north west,
  draw=white!80!black
},
tick align=outside,
x grid style={white},
xlabel={compression rate},
xmajorticks=true,
xtick style={draw=none},
xmin=0.7, xmax=7.3,
xtick style={color=white!33.3333333333333!black},
xtick={0,1,2,3,4,5,6,7,8},
xticklabels={,1\%,2\%,5\%,10\%,20\%,50\%,100\%,},
y grid style={white!93.3333333333333!black},
ylabel={seconds per epoch},
ymajorgrids,
ymajorticks=true,
ytick style={draw=none},
ymin=346, ymax=1512,
ytick style={color=white!33.3333333333333!black}
]

\path [fill=color0, fill opacity=0.2, very thin]
(axis cs:1,394.994080571972)
--(axis cs:1,384.670112207607)
--(axis cs:2,401.557716590471)
--(axis cs:3,411.985227716121)
--(axis cs:4,410.301454913171)
--(axis cs:5,419.494550377922)
--(axis cs:6,423.325225105356)
--(axis cs:7,532.779729431467)
--(axis cs:7,535.549289999917)
--(axis cs:7,535.549289999917)
--(axis cs:6,431.521942897768)
--(axis cs:5,424.466288947851)
--(axis cs:4,411.648972176981)
--(axis cs:3,422.433903950399)
--(axis cs:2,404.25287990618)
--(axis cs:1,394.994080571972)
--cycle;

\path [fill=color1, fill opacity=0.2, very thin]
(axis cs:1,660.375275012558)
--(axis cs:1,649.571725671742)
--(axis cs:2,684.202231056962)
--(axis cs:3,895.361167953714)
--(axis cs:4,1450.24732870974)
--(axis cs:4,1487.16655882353)
--(axis cs:4,1487.16655882353)
--(axis cs:3,910.1473574649)
--(axis cs:2,687.550471411229)
--(axis cs:1,660.375275012558)
--cycle;

\path [fill=color2, fill opacity=0.2, very thin]
(axis cs:1,1071.73637097018)
--(axis cs:1,987.822122059148)
--(axis cs:1,1071.73637097018)
--(axis cs:1,1071.73637097018)
--cycle;

\addplot [thick, color0, mark=*, mark size=3, mark options={solid}]
table {%
1 389.78569338716
2 402.87011214031
3 416.952379424285
4 411.028575204281
5 421.904832880723
6 427.309693358374
7 534.177028678868
};
\addlegendentry{\conveinsum}
\addplot [thick, color1, mark=*, mark size=3, mark options={solid}]
table {%
1 655.121269077282
2 685.848327625873
3 902.77300224698
4 1467.62335191017
};
\addlegendentry{\pytorch w/ ckpt}
\addplot [thick, color2, mark=*, mark size=3, mark options={solid}]
table {%
1 1029.50275621723
};
\addlegendentry{\pytorch w/o ckpt}
\end{axis}

\end{tikzpicture}
    }
\subcaption{RCP Train (temporal)}\label{fig:video-rcp-vs-pytorch-train-temporal}
\vspace{-1em}
\end{minipage}%
\hfill
\begin{minipage}[b]{.5\linewidth}
\centering
\resizebox{\textwidth}{!}{
    \begin{tikzpicture}

\definecolor{color0}{rgb}{0.886274509803922,0.290196078431373,0.2}
\definecolor{color2}{rgb}{0.203921568627451,0.541176470588235,0.741176470588235}
\definecolor{color1}{rgb}{0.596078431372549,0.556862745098039,0.835294117647059}

\begin{axis}[
axis line style={white},
legend cell align={left},
legend style={
  fill opacity=0.8,
  draw opacity=1,
  text opacity=1,
  at={(0.97, .97)},
  anchor=north east,
  draw=white!80!black
},
tick align=outside,
x grid style={white},
xlabel={compression rate},
xmajorticks=true,
xtick style={draw=none},
xmin=0.7, xmax=7.3,
xtick style={color=white!33.3333333333333!black},
xtick={0,1,2,3,4,5,6,7,8},
xticklabels={,1\%,2\%,5\%,10\%,20\%,50\%,100\%,},
y grid style={white!93.3333333333333!black},
ylabel={seconds per epoch},
ymajorgrids,
ymajorticks=true,
ytick style={draw=none},
ymin=672.920680442213, ymax=3274.58995911469,
ytick style={color=white!33.3333333333333!black}
]
\path [fill=color0, fill opacity=0.2, very thin]
(axis cs:1,811.197211734943)
--(axis cs:1,809.311383559561)
--(axis cs:2,797.680837493999)
--(axis cs:3,797.891855126853)
--(axis cs:4,791.178374927326)
--(axis cs:5,850.354460571758)
--(axis cs:6,887.096614639351)
--(axis cs:7,1361.09427982447)
--(axis cs:7,1363.19545041028)
--(axis cs:7,1363.19545041028)
--(axis cs:6,889.354224874219)
--(axis cs:5,855.87117788804)
--(axis cs:4,794.946955823782)
--(axis cs:3,801.557386907425)
--(axis cs:2,800.036214932191)
--(axis cs:1,811.197211734943)
--cycle;

\path [fill=color1, fill opacity=0.2, very thin]
(axis cs:1,1648.08090269321)
--(axis cs:1,1640.80047986953)
--(axis cs:2,1843.43954292875)
--(axis cs:3,2123.77688430585)
--(axis cs:4,2245.53010930735)
--(axis cs:4,2253.40970882798)
--(axis cs:4,2253.40970882798)
--(axis cs:3,2132.33727578151)
--(axis cs:2,1855.20582699139)
--(axis cs:1,1648.08090269321)
--cycle;

\path [fill=color2, fill opacity=0.2, very thin]
(axis cs:1,3156.33226462958)
--(axis cs:1,3137.35956281991)
--(axis cs:1,3156.33226462958)
--(axis cs:1,3156.33226462958)
--cycle;

\addplot [thick, color0, mark=*, mark size=3, mark options={solid}]
table {%
1 810.286726909255
2 798.830193285604
3 799.659436504972
4 792.972112924258
5 853.007709490447
6 888.249353679901
7 1362.07882422257
};
\addlegendentry{\conveinsum}
\addplot [thick, color1, mark=*, mark size=3, mark options={solid}]
table {%
1 1644.46058021686
2 1849.39908427744
3 2127.90901954
4 2249.39714615065
};
\addlegendentry{\pytorch w/ ckpt}
\addplot [thick, color2, mark=*, mark size=3, mark options={solid}]
table {%
1 3147.53106714877
};
\addlegendentry{\pytorch  w/o ckpt}
\end{axis}

\end{tikzpicture}
    }
\subcaption{RCP Test (temporal)}\label{fig:video-rcp-vs-pytorch-test-temporal}
\vspace{-1em}
\end{minipage}
\caption{\textbf{Run-time comparison between \conveinsum and \pytorch for a video classification machine learning task} An RCP-TNN ($R=3$) is trained on the UCF-101 dataset. All tests were run using the maximum allowable batch size. \pytorch w/ checkpointing was only able to run without memory overflow for compression rates 1\% - 10\% and \pytorch w/o checkpointing was only able to run without memory overflow for compression rate 1\%. Runtimes are averaged over 3 random runs, and error bars are denoted by the shaded areas. }\label{fig:video-rcp-vs-pytorch}
\vspace{-1em}
\end{figure}